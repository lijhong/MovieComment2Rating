\begin{abstract}
近年来,随着各种论坛、评论网站不断涌现,越来越多的用户文本出现在网络上。这些文本中很多具有情感倾向,如影评、商品评论、博客评论等。分析这种倾向,有利于获得大众对于电影或商品的评价、了解舆情。因此,对文本进行情感分析的研究有着很强的实际意义。

目前对文本进行情感分析的方法有很多,传统的方法有基于知识的方法,如使用情感词典;也有基于机器学习的方法,如朴素贝叶斯和支撑向量机。近年来,深度学习的方法也被用在文本情感分析的问题上。卷积神经网络(Convolutional Neural Network,简称为CNN)是一种发源于计算机视觉领域的深度学习算法,因其模型非常适合图像数据的特征而在计算机视觉领域取得了巨大成功。通过使用“词嵌入”(word embedding)的方法,卷积神经网络也被应用于文本领域,在文本分类等问题上取得了成功。

本文以来自豆瓣的中文短评为语料,使用卷积神经网络对其进行情感分析,判断其情感倾向,并由此预测每一条短评对应的评分(星数)。本文采用Word2vec的算法,从分词后的16万句短评和10万篇长评的语料训练出一个中文词语的word embedding。基于训练出的word embedding 将每个词语转化为词向量,在此基础上搭建神经网络的模型,进行情感分析。本文重点实现了卷积神经网络的多种结构,同时实现了复发神经网络(Recurrent Neural Networ,简称RNN),以及传统算法如朴素贝叶斯和支撑向量机在这一问题上的应用。分析和比较了不同算法的实验结果以及性能。

\keywords{情感分析,深度学习,卷积神经网络}

\end{abstract}

\begin{enabstract}
Many kinds of forums and comment sites appears in recent year, therefore, more and more user generated text content can be found on the internet. A big part of text content generated by user 

\enkeywords{University of Science and Technology of China (USTC), Thesis, Universal \LaTeX{} Template, Bachelor, Master, PhD}
\end{enabstract}
