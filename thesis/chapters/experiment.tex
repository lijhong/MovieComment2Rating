\chapter{实验结果与分析}
\section{实验结果}
评分预测(五分类实验)实验结果如下:

【表格】

二分类预测实验结果如下:

【表格】

\section{结果分析}
可以看到使用中文分词器分词的效果要好于按字断句,说明对中文语言文本分词的确可以提取出文本中很多的特征。而基于字断句的话,在训练数据不非常大的情况下(40000句仍然只能算比较少的数据量),或是模型复杂程度不够高的情况下,可能会导致一些信息或特征无法被提取出来,因而导致结果不如分词。但同时也可以注意到,在卷积核的设置包含了3或4这样比较大的窗口时,基于字断句的输入也会有很好的效果。这可能说明了说明卷积神经网络的确有提取局部特征(在这里体现为提取出字的组合的特征)的能力。【可能通过maxpooling的结果补一个具体例子】

单层双通道的模型也取得了很好的实验结果,但其最好的结果没有超过单层单通道的最好实验结果。这与我们开始时的期望略有出入,可能是由于双通道的模型更容易过拟合。双层卷积层模型的结果也超过了baseline算法,但落后于单层卷积层的两种模型。这与图像数据领域的结果有很大差别,可能是因为文本数据每个单词相较于图像数据的像素,包含的信息要多得多。因此,即使在单层卷积层的模型也可以提取出很多的信息,也会有很好的结果。而多层之后,反而由于抽象的层数太高、过拟合的机会增大,可能会导致效果反而不如单层的网络。

另外也可以注意到,五分类问题的结果准确率普遍比较低,不超过45\%。这是由于对影评进行五分类本身是一个比较困难的问题,与本课题数据集相近的SST-1数据集(同样是影评数据,分为五类,分别为非常负面、负面、中立、正面和非常正面),目前最前沿的成果也只能达到48.7\%的准确率。这个问题准确率难以踢得很高是因为每个影评者自己的指标不同,同样一句影评,对于一些人来说会选择给四星,另一些人则会给五星、三星。此外,中间三档影评的区分度很低。本课题使用的数据集未经过人工筛选,只根据作者发过的影评数设置了一个阈值。故数据集中可能存在评论与打分不符,或是一些没有意义的垃圾信息。这些评论也会对结果产生影响。

【表格】confusion matrix

上表是五分类问题最后结果的confusion matrix,即预测错误的分布。可以看到,大多数的错误分布在祝对角线上,即出现在相邻的两个分段。尤其是中间二、三、四三个阶段,出现的错误非常多。这印证了我们上一段的分析。

因为五分类的结果准确率较低,难以应用在实际问题中。我们又增加了二分类的试验,可以看到,二分类实验的准确率很高,达到了82\%以上。

\section{小结}
本章对实验结果进行了总结和分析。

